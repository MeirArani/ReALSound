% Preamble
% ---
\documentclass{report}
% Packages
% ---
\usepackage{amsmath} % Advanced math typesetting
\usepackage[utf8]{inputenc} % Unicode support (Umlauts etc.)
\usepackage{hyperref} % Add a link to your document
\usepackage{graphicx} % Add pictures to your document
\usepackage{listings} % Source code formatting and highlighting
% ---
%
\begin{document}


%%% TITLE

\author{Meir Arani \\ Kyushu University \\ Graduate School of Design} % The authors name
\title{ReAL Sound: Outline of a Reusable Audification Library to Improve Game Accessibility for the Visually Impaired} % The title of the document
\date{\today{}} 
\maketitle{} % Generates title

%%% ABSTRACT

\begin{abstract}
    In a world of ever-increasing software complexity, there has been a growing demand for interoperable, reusable technologies that function in many problem domains. This is especially true in the world of game development, where tools, structures, and architectures often change from title to title. At the same time, the specificity of software user needs has also grown immensely, bringing an increased demand for advanced accessibility tools with it. To address these trends, we propose ReAL Sound: the ReUsable Audification Library, which abstracts the creation of visual accessibility technology for impaired persons in the realm of game design using computer vision and machine learning techniques.  
\end{abstract}

\newpage{} % Pagebreak

%%% TABLE OF CONTENTS

\tableofcontents{} % Generates table of contents from sections and subsections
\newpage{} % Pagebreak


%%% INTRODUCTION

\chapter{Introduction}
\section{Game Development}
\section{Accessibility}
\section{Computer Vision}
\section{Machine Learning}


%%% LITERATURE REVIEW
\chapter{Literature Review} 

%%% PROPOSAL

\chapter{Real Sound}
\section{Proposal} 
\section{Outline}
\subsection{Design}
\subsection{Training}
\subsection{Implementation}
\subsection{Play}
\section{Considerations}


%%% IMPLEMENTATION

\chapter{Sample Implementation}
\section{OpenCV}
\section{Qt}

%%% EXPERIMENTS

\chapter{Experiments} 
\section{Pong Demonstration}
\subsection{Results}

%%% CONCLUSION
\chapter{Conclusions}
\section{Limitations}
\section{Future Work}

\section{Thanks}


\end{document}
% Preamble
% ---
\documentclass{report}
% Packages
% ---
\usepackage{amsmath} % Advanced math typesetting
\usepackage[utf8]{inputenc} % Unicode support (Umlauts etc.)
\usepackage{hyperref} % Add a link to your document
\usepackage{graphicx} % Add pictures to your document
\usepackage{listings} % Source code formatting and highlighting
\usepackage[nottoc,numbib]{tocbibind}
\usepackage{xspace}
\usepackage{biblatex}
\usepackage{amsthm}
\usepackage[]{algorithm2e}
\usepackage{epigraph}
\usepackage{framed, blindtext}

%%%%%%%%%%%%%%%%
\newcommand\lorname{\mdseries\textit{List of recommendations:}}%list of recommendations
\newcommand\lorext{lor} % another aux-file is needed
\newcommand\lortmp{}
\newcounter{recommendations}
\newcommand{\Summary}[2][\empty]{%
  \vspace{0\baselineskip} \begin{framed} #2 \end{framed}%
  \ifx#1\empty\renewcommand\lortmp{\footnotesize #2 }\else\renewcommand\lortmp{\footnotesize #1}\fi%
  \addcontentsline{\lorext}{subsection}{\therecommendations\quad\lortmp}
}
\makeatletter
  \newcommand\listofrecommendations{%
    \footnotesize
    \section*{\lorname}
    \markboth{\lorname}{}
    \@starttoc{\lorext}
  }
\makeatother


\addbibresource{references.bib}


\newcommand{\rs}{ReAL Sound\xspace}
\newcommand{\rsfull}{\textbf{Re}-usable \textbf{A}udification \textbf{L}ibrary\xspace}
\newcommand{\plan}{\textbf{Planning}\xspace}
\newcommand{\train}{\textbf{Training}\xspace}
\newcommand{\design}{\textbf{Design}\xspace}
\newcommand{\exec}{\textbf{Execution}\xspace}
\newcommand{\state}[1]{\textbf{#1}}
\newcommand{\rules}{\textbf{Rules}\xspace}
\newcommand{\vision}{\textbf{Vision}\xspace}
\newcommand{\decision}{\textbf{Decision}\xspace}
\newcommand{\audio}{\textbf{Audification}\xspace}

\newcommand{\intro}{\subsubsection{Introduction}}
\newcommand{\conclusion}{\subsubsection{Conclusion}}


\newcommand{\imp}{implementor\xspace}


% ---
%
\begin{document}

%%% TITLE

\author{Meir Arani \\ Kyushu University \\ Graduate School of Design} % The authors name
\title{ReAL Sound: Outline of a Reusable Audification Library to Improve Game Accessibility for the Visually Impaired} 
\date{\today{}} 
\maketitle{} 

%%% ABSTRACT

\begin{abstract}
    In a world of ever-increasing software complexity, there has been a growing demand for interoperable, reusable technologies that function in many problem domains. This is especially true in the world of game development, where tools, structures, and architectures often change from title to title. At the same time, the specificity of software user needs has also grown immensely, bringing an increased demand for advanced accessibility tools with it. To address these trends, we propose ReAL Sound: the ReUsable Audification Library, which abstracts the creation of visual accessibility technology for impaired persons in the realm of game design using computer vision and machine learning techniques.  
\end{abstract}

\newpage{} % Pagebreak

%%% TABLE OF CONTENTS

\tableofcontents{} % Generates table of contents from sections and subsections
\newpage{} % Pagebreak


%%% INTRODUCTION

\chapter{Introduction}
\section{Recent Advances in Software Development}
\section{Recent User Experience Trends}
\section{Games and Accessibility}


%%% LITERATURE REVIEW
\chapter{Literature Review} 
\section{Computer Vision}
\section{AI \& Machine Learning}
\section{Audification}
\subsection{Spatial Audio}
\section{Games}
\subsection{Games and Accessibility}
\subsection{Games and Audification}
\subsection{Games and Computer Vision}

%%% PROPOSAL

\chapter{ReAL Sound}
\paragraph{Introduction}
In this chapter, I propose and outline the \textbf{\emph{\rs}} system: a \rsfull that abstracts the design of accessibility features for the visually impaired via the use of computer vision, spatial audio, and audification techniques.

\section{Proposal} 
\paragraph{Introduction} 

In this section, I propose \rs and justify its novelty and utility based on the literature review performed in the previous chapter.

\subsubsection{\emph{What is \rs?}}
\rs (\rsfull) is a software framework concept that generalizes the creation of visual accessibility features for games. Using computer vision techniques and modern spatial audio technology, \rs aims to abstract the process of moment-to-moment analysis of a game's state as well as the generation of 3D spatial audio objects. Through this abstraction, implementors of \rs (as distinguished from \emph{users} of \rs---the visually impaired persons benefitting from the features) can sidestep many of the language, platform, and architecture specific headaches involved in the creation of game accessibility features, especially in the case of after-market games. 

In essence, \rs seeks to convert a game's visual information (the things a player sees on-screen) into audio information (the things a player hears). This is done so that visually impaired persons can better understand a game without having to 'look' at it directly. Moreover, \rs aims to abstract this process, making a simple implementation possible for a plethora of games. \rs achieves this generalization by abstracting feature development process into a few core steps for any given target game:

\begin{description}
    \item [Planning] The creation of simple 'rules' and 'definitions' which accurately describe the game.
    \item [Training] The act of preparing computer vision techniques based on the \textbf{Planning} stage to analyze the game's current state data.
    \item [Design] The creation of simple 'rules' and 'definitions' for translating the data analyzed in the \textbf{Training} phase into spatial audio objects. 
    \item [Execution] The real-time marriage of the previous stages. Uses a computer vision model \textbf{trained} on the \textbf{planned} rules which translates the game's real-time visual data into spatial audio objects as \textbf{designed} by the implementor.    
\end{description}

With \rs, a implementor with only a moderate amount of technical knowledge  could feasibly add accessibility features to any target game---all with zero knowledge of the game's underlying source code or architecture. A visually impaired user of \rs, meanwhile, gains access to accessibility features which translate the game's visual information into easily parsable audio data---allowing them to intuitively understand a game without seeing the game's visual output. 


\subsubsection{\emph{Why is \rs?}}
As discussed in the previous chapter, there is a well-established and ever-growing need for software interoperability and platform agnostic support, as well as a growing demand for improved software accessibility. Implementors of game accessibility features already face numerous game-specific challenges in the design process---as each game's bespoke nature demands equally unique accessibility design. The added barriers of specific architecture, engine, and language implementation have especially stymied the development of accessibility features in the gaming industry---where many development tools target a single architecture or operating system, are company specific, and are often used on a per-game basis. 

\rs significantly streamlines these problems through abstraction---significantly simplifying this troubled implementation process. As a consequence, greater flexibility and more time are handed to feature designers---who are now better quipped to craft higher quality features at a faster clip.

Moreover, the generalization process requires zero knowledge of the game's actual codebase, meaning games can be modified in the after-market by dedicated enthusiasts. As a consequence, the fan-driven movement of retrofitting of older titles with modern accessibility features becomes easier---allowing visually impaired persons access to a wealth of classic titles while costing modern developers zero resources. This is becoming especially crucial in modern times, as some estimates show that over 85\% of games are functionally abandoned---with no publisher or developer entity maintaining ownership \cite{Lewin_2023}. This fact effectively renders the vast majority of published games, \emph{'abandonware'} with little hope of official improvements by original developers \cite{Costikyan_2000}.

\subsubsection{\emph{How is \rs?}}
\rs is made possible through modern machine learning, computer vision, and spatial audio technologies. Using the latest computer vision techniques through libraries like OpenCV\cite{opencv_library}, even novice developers can implement object detection algorithms---which analyze visual input and return semantic information about the its contents---with ease. 

\paragraph{The Modern AI Boom}

On top of this, the recent 'boom' in AI technologies\cite{Knight_2023}\cite{Meredith_2023} has brought an equally intense focus to the development, improvement, and democratization of AI tools and systems\cite{Brynjolfsson_McAfee_2017}\cite{Heikkilä_2022}. Ecosystems like \emph{HuggingFace}\cite{Wiggers_2022} and projects like Google's \emph{TensorFlow}\cite{tensorflow2015-whitepaper} have dramatically changed the landscape of AI technology development---trivializing many tasks considered inaccessible to the common developer just a few years ago.

\subparagraph{Agnostic Vision Requirements}

Considering the ever-changing state of modern AI and computer vision technology, \rs does not specifically call for any one particular solution for its \textbf{Training} or \textbf{Execution} stages. Instead, \rs only requires the \emph{successful} real-time identification of in-game objects \textit{as defined by the implementor}. The means by which this goal is accomplished is left up to the implementor, and may be achieved by any available means. 

Consequently, \rs does not technically demand the usage of AI \emph{at all.} Many well-proven pre-AI techniques---such as \emph{template matching}\cite{Brunelli2009-gh} and \emph{corner detection}\cite{Hildreth1980}---have proven to be successful in many simple use-cases. I will demonstrate this in a later chapter, creating one  implementation of \rs using the original corner detection algorithm---the \emph{Harris corner detector}\cite{Harris}, originally developed in 1986. 

\paragraph{Spatial Audio}
Running alongside the AI boom has been a comparably smaller (yet not insignificant) boom of 3D and spatial audio technologies. As discussed in the previous chapter, many advances have been made in sonification, audification, and spatial audio techniques in recent years. This trend has given end-users an abundance of choice for low-cost, high quality audio playback devices with native spatial audio support--from in-ear monitors like the the Apple \emph{AirPods}\cite{ApplePods} to even computer display monitors from major manufacturers like Dell\cite{Campbell_2025}. Future patents promise even greater advancements through advanced techniques like automatic HRTF adjustments specific to the user's unique physiology\cite{Antti2024}.


Many new software development libraries been developed to address this boom in consumer demand. Fan driven efforts like the \emph{Spatial\_Audio\_Framework}\cite{McCormack2024} as well as company-produced projects such as Valve Software's Steam Audio\cite{Valve} have seen popular adoption. Legacy libraries with widespread adoption, such as the quarter-century old \emph{Qt} application development framework\cite{QTProgramming}, have also introduced support for modern spatial audio technologies in recent versions\cite{QTSpatial}. The implementations of \rs presented later in later sections utilize Qt as a backend framework---although numerous other libraries feature equivalent functionality. 

\paragraph{Conclusion}
In this section, I proposed \rs and provided justifications for its utility and novelty by considering contemporary technology innovations and emergent user trends. 





\section{Concepts}
\paragraph{Introduction} 
In this section, I explain in detail the theoretical concepts which underpin \rs. I later use these concepts to rigidly define the system's core structure. To begin, I use formal math notation to construct a semi-formal definition of a 'video game.' I then explore this definition through the lens of automata theory. Following this, I use the construct to abstract methods of interpreting a game via visual analysis. 


\subsection{Anatomy of a Game}
\subsubsection{Introduction}
Here, I construct a semi-rigid abstraction of video games using formal notation. Later, I will use these constructs to more clearly explore related concepts.


\subsubsection{Game}
A Game, termed $M$ (for \emph{meta}), can be conceived as the combination of three sets: a collection of meta attributes $A$, a series of in-game states $G$, and a group of entities $I$:

$$
M = \{A, G, I\}
$$

\paragraph{Meta Attributes}
The game's meta-attributes $M_A$ can be imagined as data that is preserved over an entire game session. This data can persist between state transitions and usually describes overarching information such as current playtime or the currently active game level. 

\subsubsection{Game State}
Each state, $G_S$ (referred to interchangeably as \textit{game state} hereafter) is comprised of four components---internal state-attributes $A$, internal state-logic $L$, as well as conditions $C$ for inter-state transitions $T$.

$$
G_S = \{ A, L, C, T \}
$$

\paragraph{State Attributes}
A state's attributes $A$ act similarly to the game's overarching meta attributes $M_A$, but on a per-state context. These attributes may store information such as the current time spent within the state or other data describing the state in specific. State data is lost upon transitioning to a different state.

\paragraph{Internal Logic}
The state's internal logic $L$ defines all the behaviors and activities that are carried out \textit{within} the state. For example, a racing game may contain a \textbf{start} state---which contains logic for playing a special sound effect when the race is started, or routines to visually display the text \textit{'START!'} on-screen. 

In abstract, $L$ can be imagined as a series of conditional requirements ('rules') $L_C$ that yield specific game actions ('responses') $L_A$:

$$L: L_C \longrightarrow L_A$$


\paragraph{Conditions}
A state also contains conditional rules defining when to exit the state and transition to different state. $C$ defines these rules as a set of boolean statements---which evaluate to either \texttt{true} or \texttt{false}.
Take, for example, a sports game which transitions from a \textbf{match} state to a \textbf{finish} state after ten points are scored in the match. Imagine that the current match score is stored as an integer value through the state attribute $S$: 

$$
S: [0, 10],  S \in G_A
$$

Then there is a condition within $C$, lets call it $C_S$, which might look like this:

$$
C_S = S\geq10 
$$

When $C_S$ evaluates to true (i.e., when the match score has reached ten points), the state will transition transition to the \textbf{finish} state.

\paragraph{Transitions}
The transition function $T$ defines the mapping of "where" to or "how" to transition to a different state after a condition in $C$ has evaluated to \texttt{true}. In essence, $T$ maps a conditional statement $C_S$ to the next game state $G_N$ to transition to:

$$
T: C_S \longrightarrow G_N
$$

This function can be generalized by instead taking a specific state $G_S$ and a generic condition $C$ as input:

$$
T: G_S \times C \longrightarrow G_N
$$

\subsubsection{Entities}
Entities $I$ are the objects which constitute a game's internal structure---the 'actors' of the game. An entity can be anything contained within a game, such as the player, enemies, items, level construction, etc.  Usually, an entity's behavior can be described using verbs---in either an active (\emph{Mario jumps on the platform}) or passive (\emph{The platform was jumped on by Mario}) sense.

Each entity has attributes $A$ as well a collection of states $E$:

$$I = \{A, E\}$$

\paragraph{Entity Attributes}
Similar to the previous definitions, entity attributes describe entity-specific attributes. For example, an entity's location point $P$ in a 2D game's world space might be modeled as:

$$P = \{(x,y): x,y \in \mathbf{R} ^2\}$$ 

Similar information, like a enemy's current health, an item's purchase price, or a bomb's damage radius are also considered Entity attributes. 

\paragraph{Entity States}
An entity state, then, describes an entity's different states of being. A player entity might be able to \textbf{jump}, while a spell entity might be \textbf{cast} or an enemy entity might be \textbf{killed}, to name a few examples. 

Some examples may seem unintuitive, but still work within this framework. A game's menu system, for instance can also be considered an entity. In this case, the act of clicking on the menu---triggering some sort of effect, can be modeled as a \textbf{clicked} state with its own internal logic and goals. 

As you might guess, a entity state looks similar to a game state, with its own internal logic $L$, a series of transition conditionals $C$, and transition mappings $T$:

$$E = \{L, C, T\}$$

One example of an entity state is \textbf{jumping}, which may transition from \textbf{idle} or \textbf{walking} or \textbf{running}. When transitioned to, the \textbf{jumping} state might, according to $L$, play a jumping sound effect. When a condition in $C$ is satisfied (the jumping entity finally touches the ground again, or perhaps falls into a pit and dies), then the transition function $T$ moves to the next state $E_N$:

$$T(C) \longrightarrow E_N$$

\subsubsection{Summary}
To summarize, a game $M$ can be conceptualized as:

\begin{enumerate}
    \item A collection of attributions ($A$), which describe aspects of the game's overarching meta-state.
    \item A set of game states ($G$), which each contain internal game-logic ($L$), state-specific attributes ($A$), as well as rules for when to transition to another state ($C$) and exactly which state to transition to ($T$).   
    \item A set of entities ($I$) which serve as the game's passive and active subjects, each having their own set of attributes ($A$) and states ($E$). Each state compasses the same qualities of a state enumerated above ($A$, $L$, $C$, and $T$).   
\end{enumerate}

$$
M = \{A, G, I\}
$$

$$
G_S = \{ A, L, C, T \}
$$

$$I = \{A, E\}$$

$$E = \{L, C, T\}$$


\subsubsection{Limitations}
While this construction is useful for our purposes here, I do not contend this definition as a end-all definition of games. It is already known that several games---both in physical and video formats---are Turing complete, which exists outside the definitions given here. Previous researchers however have likewise made attempts in formally defining games with similar bounds on their definitions---leaving titles like \emph{Magic: The Gathering}\cite{churchill2019magicgatheringturingcomplete} or \emph{Minecraft}\cite{Hildreth1980} out of scope\cite{Demaine}. Virtually any game can also be made Turning complete if methods of arbitrary code execution (ACE) are discovered, which has been demonstrated using \emph{Super Metroid}\cite{Mawhorter_Aytemiz_Karth_Smith_2021}, \emph{The Legend of Zelda: Ocarina of Time}\cite{OoT}, and numerous others. Despite this, we find that our definition of games covers a sufficiently useful scope to demonstrate \rs's utility.


\subsubsection{Conclusion}
Here, I have provided a general outline of games in formal notation. I do not guarantee nor contend that these structures form a complete closure over the entire concept of video games---whose definition is still fiercely debated to this day\cite{Knoop_2021}---but these structures will serve as a useful framework for our following definitions and concepts.

\subsection{State Models}

\subsubsection{Introduction} 
In this section, I briefly review automata theory theory. Through concepts like finite state automata, I produce the Game and Entity (GSM and ESM) state machine constructs, which I will continue to utilize in following sections.

\subsubsection{Automata Theory}
The previous section's underlying theory will likely seem familiar for those with a background in fields such at theory of computation or advanced linguistics. I am, of course, referring to the theory of \emph{automata} a core concept which underpins numerous fields--including computing and, by natural extension, interactive software such as video games. 

Automata are generally categorized into four major types as defined by the Chomsky Hierarchy---ranging from Type-0, describing the classical Turning machine, to Type-3, which describes finite state automata \cite{Chomsky1956ThreeMF}. Games, as I have described them above, can (in general) be described as a Type-3 automata---an automation of finite state and quality. 

\subsubsection{Finite State Automata}

\paragraph{Introduction and Concepts}

Carroll defines (deterministic) finite automata as the "mathematical model of a machine that accepts a particular set of words over some alphabet $\Sigma$.\cite{FSM}" He conceptualizes FSMs as "black boxes" that combine an input tape (consisting of symbols in the alpha $\Sigma$), a \emph{'read head'} which processes this input tape of symbols, and an 'acceptance light' which indicates the acceptance/rejection of an input symbol, and whose activity is governed by the read head's reaction to the given input. In essence, the machine accepts input, makes decisions for the acceptance light based upon the input, and then moves to a new position to receive another input. The machine's acceptance/rejection of a symbol, as well as the position it moves to next, is contingent on its own internal logic---some rule that dictates \emph{"If symbol X is read, turn on (if accepted) or off (if rejected) the acceptance light, and move read head to position Y."} The machine may also read a symbol which finally halts its operation.

\paragraph{Comparison with Turing Machines}

This conceptual structure bares some resemblance to a Turning machine\cite{Turing}, although there are some distinct differences. A Turning machine has an infinite amount of memory and is capable of accepting languages with recursive qualities---as the turing machine is able to 'write' to tape--modifying a symbol after reading it. A Turing machine is well known to be capable of implementing and computing any possible computing algorithm\cite{Turing_2004}. It is consequently classified as a Type-0 (also known as a "recursively enumerable" or an "unrestricted") grammar.

An FSM, by contrast, is limited in several ways. As the name implies, FSM have a \emph{finite} set of states. More importantly, the system is incapable of storing memory---lacking the 'writing' mechanism of a Turning machine\cite{FSM}. 

\paragraph{Deterministic and Non-deterministic FSAs}

The FSMs we detail here should also be distinguished as \emph{deterministic}, as opposed to \emph{non-deterministic} FSMs. Some FSMs are considered non-deterministic when their states are capable of transitioning to several possible outputs given the \emph{same input}---making their reaction to an input non-deterministic in nature\cite{NDFSA}. For our purposes, we define games as a type of deterministic FSM--where the same inputs always yield the same results. 

\paragraph{Obstacles and Limitations}
You may be wondering how the provided definition of games fit the category of FSM considering the inclusion of attributes $A$---which may persist as a game or entity transitions from state-to-state. This may appear like a kind of 'memory' on first glance, but is actually just a shorthand notation that simplifies our FSM conceptualization. In reality the proper FSM of a game has \emph{many} more states than we describe here. 

To provide some insight into this concept, imagine that for each state $G_S$, the conditions of $C$ and the transitional rules of $T$ also encode the attributes of $A$. This means that each state $G_S$ actually has numerous variations---each relating to the specific attributes of $A$ and the conditions of $C$. For example, the aforementioned sport game's \state{match} state may really have $10$ different \state{match} states, one describing each possible value of the scoring attribute $S$. If the game has just begun, the first scoring condition $C_0$ will transition the machine to $match_1$, and so on. Eventually, the final \state{match} state's $C_9$ condition will transition the FSM to the \state{finish} state.


\paragraph{Useful Adaptions}

As you can see, this more-accurate representation of FSMs is considered unproductive in many domains where FSMs remain useful concepts\cite{state}. Thankfully, numerous alternative notations have arisen to address these shortcomings. 

For example, FSMs which produce output contingent on a given input---often termed \emph{transducers}\cite{FSM}---are frequently conceptualized as \emph{Mealy machines}. The output of a Mealy machine state is also contingent on its input (\texttt{IF input(X) AND IN state(Y) THEN transition(Z)}). Later designs---such as the \emph{unified modeling language}---have extended this notation further to leverage the theoretical benefits of FSAs while avoiding their limiting notation schemes\cite{omg2011umls}. In a similar fashion, I make use of FSAs on a conceptual basis while abstracting away these notational hindrances through the usage of the attributes $A$ of each object. 

\subsubsection{Automata and Games}
It is possible to define games more clearly through their relation to FSAs using our definitions. For example we can consider a game to be a collection of two principal state machines: 
\begin{description}
    \item[Game State Machine (GSM)] The game state machine comprises the general 'flow' of a game---its levels, combat encounters, win conditions, story sequences, game-over menus, boss fights, item inventories, and any other 'macro' loop that is reasonably distinguishable within the gaming experience.
    \item[Entity State Machine (ESM)] An entity state machine comprises the general 'life' and 'activities' of a given entity---every action it can take, as well as its response to any given input. A player may run, jump, or die, while a stage platform may move or vanish.     
\end{description}

These two concepts---the GSM and ESM---prove to be crucial in the conceptualization of \rs. We will continue to analyze them in later sections using other frames of analysis.

\subsubsection{Conclusion}
In this section I reviewed numerous aspects of automata theory---namely finite state automata---in order to present the concept of the Game and Entity state machines (GSM and ESM), which will prove to be useful constructs in the total construction of \rs.


\subsection{How Players Understand Games}

\subsubsection{Introduction}
In this section, I explore how games are conceived from the player's perspective. I illustrate my points via a few examples and ultimately generate the concept of the \state{State-Rule} relationship, which has consequences for \rs's design.
 
\subsubsection{How Players Understand Games}

\paragraph{Introduction}

Drifting slightly from the previous formal definitions, let us consider here exactly how a human player understands a game. To begin, let us consider a simple game familiar to most readers---the playground classic \emph{rock-paper-scissors}. 

\subparagraph{Simple Case: \emph{Rock-Paper-Scissors}}

In rock-paper-scissors, each player is allowed one input---the 'hand' they choose to play---and the game produces one 'output'---the outcome of the match. Experienced players have an intuitive understanding of the rules which govern the game, which are simple enough to enumerate here:

\begin{description}
    \item [Rule One:] Each Player produces a hand symbol, which may be one of:
        \begin{enumerate}
            \item \state{Rock}
            \item \state{Paper}
            \item \state{Scissors}
        \end{enumerate}
    \item [Rule Two:] The outcome of the match is decided by these stipulations:
        \begin{enumerate}
            \item \state{Paper} 'beats' \state{Rock}
            \item \state{Rock} 'beats' \state{Scissors}
            \item \state{Scissors} 'beats' \state{Paper}
            \item If both players produce the same hand, the game is a draw
        \end{enumerate}
\end{description}

These rules comprise the entire experience, allowing a full play of the game to occur over the course of mere moments.

To use the constructs defined previously, it is clear that rock-paper-scissors (hereafter \emph{RPS}) has only one possible game state, a fact which highlights the game's simplicity. In other words, players require no greater sense of \emph{context} to play RPS---they only need to understand the base rules described above. Consequently, the game has no memory requirements, as each play is decided in one step and has no consequences on future or past matches.

\subparagraph{Average Case: \emph{Rock-Paper-Subsequent}}
The previous definition of RPS can seem trivial, leading to an unfulfilling experience for gamers. As you likely know, many performances of RPS actually append on additional rules to improve the play experience. For example, competitions often involve some \emph{'best-of'} requirement---stipulating that a player must win a certain amount of matches before being declared the 'overall' winner. 

This example of RPS, which I will term \emph{Rock-Paper-Subsequent}, has added additional complexity to the game's structure. The logic of each round is now contingent on the last, requiring players use a sense of context and memory to successfully complete the game. For example, forgetful players competing in a 'best-of-one-thousand' variant of RPS may eventually lose-track of the game's score. This mistake would render the entire competition invalid, as it makes the question of who will win the match undecidable. 

In automata terms, \emph{Subsequent} has gained a sense of state---one for each possible configuration of game score. This case can be described in terms of an FSA, and, I conjecture, is categorically equivalent to my definition of games above.

\subparagraph{Advanced Case: \emph{Rock-Paper-Scheherazade}}
Imagine now that instead of having one simple match or even a simple 'best-of', we created a new version of RPS named \emph{Rock-Paper-Scheherazade}. In this version of the game, a tied match (where players produce the same hand) requires that players perform a 'best-of-three' sub-match of the game. Whichever player wins this sub-match will 'win' the match. This logic also applies recursively, meaning that players who tie during a sub-match are forced to perform yet another sub-match \emph{within} the sub-match---a sub-sub-match, if you will. This logic can naturally extend to an infinite recursion of context and memory: \emph{sub-sub-sub-sub-sub...-matches} are possible in \emph{Scheherazade}, each requiring the player to keep track of the game's current 'match', as well as all of the matches that exist 'above' it.

This rendition of RPS is, in some senses, comparable to a Turing machine---allowing for infinite memory and recursive processes---which extends beyond the definition of games we contend with in this paper. In this cases, RPS may have an infinite amount of game states---each contextualized by the particular score of the current sub-match as well as all of the sub-matches that exist in the levels above it. 

The purpose of these examples was to illustrate in clear terms the two key parts of a game's structure from the player's perspective---\state{Rules}, and \state{States}.

\paragraph{Rules}

Rules constitute the bulk of a player's understanding of a game. Without rules, even simple titles like RPS are rendered non-deterministic---as an infinite amount of hands outside of \state{Rock}, \state{Paper}, and \state{Scissors} may be produced, each with arbitrary rules about which hand 'beats' which, which may also change on a match-to-match basis. In general, players expect a specific reaction to a given input. 

It should be noted that this is true even in cases where the output is seemingly random to the player. For example, a player always expects to be dealt a set of playing cards when performing an ante in Poker, even if those cards are given to the player in a seemingly random sequence. The player is not, by contrast, expecting to receive a collection of random objects that \emph{differ} each ante in lieu of playing cards---a gun for one hand, a hot-dog for the next, and perhaps the entirety Shakespeare's \emph{Hamlet} the last. 

\paragraph{States}

Despite this, rules are not the end-all-be-all of understanding most games---which are usually comprised of many different states. In these cases, the rules of the game actually best described as the \emph{rules of the state.} This means that the same rule, applied in different states---may actually produce a different output given the same input. This provides rules with a greater sense of context that games often demand. 

\paragraph{The State-Rule Relationship}

Consequently, one may imagine the intertwining of these two concepts: \state{Rules} which define \state{State}, and \state{States} which identify \state{Rules}. This \state{State-Rule} relationship often relates to problems in \emph{context sensitivity}, a domain which extends far beyond the realm of video games. Understanding this relationship, as well as context sensitivity, is key to \rs' design. We will now explore the problem further through a well known example: \emph{The Legend of Zelda: Ocarina of Time}, which pioneered advanced solutions to context-sensitive game design.

\subparagraph{Context Sensitivity and \emph{Ocarina of Time}}

As three-dimensional games saw increasing feasibility in the early-to-mid 1990s, developers were suddenly given an extra degree of spatial freedom. Many designers quickly came to view this boon as a bane, as developers new to 3D struggled in translating these newfound liberties into sensible game controls---which are limited by both physical design and human physiology. In essence, designers often believed they needed twenty buttons to accomplish in 3D space what was once possible with only four buttons in 2D. This was a consequence of the exponential growth of \state{States} and the \state{Rules} which governed 3D experiences. 

\emph{The Legend of Zelda: Ocarina of Time} effectively solved this problem by standardizing a clear sense of context-sensitive design---both in terms of control design and UI design. In \emph{Ocarina}, each button on a game controller may have more than one in-game action (a control) assigned to it. Exactly which action is triggered in game is dependant on the game's current state. In this sense, the specific \state{Rule} of what happens when a button, say \state{A} is pressed is unique to the current \state{State} the game is in. For example, the \state{A} button may allow the player to 'talk' to another character---\emph{if} they are standing near one. They may also be able to push a box, open a door, read a sign, or use a fishing rod---all from just pressing the \state{A} button, who's \state{Rule} is contextually dependant on the current \state{State}.This was also made clear through the game's novel UI design--where the contextual action of the button (the button's \state{Rule} in the current \state{State}) was clearly displayed on-screen at all times. 

This marriage context-sensitive design with clear visual indicators of the \state{State-Rule} relationship significantly aided players---who were still new to the concept of 3D games and their control schemes---and cemented \emph{Ocarina}'s status as a model for modern game design. In essence, these choices clarified the relationship of \state{Rules} and \state{States} to players. I conjecture that an obfuscation of this relationship (on the part of the developer) or a lack of understanding of this relationship (on the part of the player) inhibits the completion---or, at the very least, \emph{the enjoyment} of the play experience. 

\subsubsection{Conclusion}
In this section, I elaborated on the experience of playing a game from the player's perspective, making use of the structures defined in previous sections. Here, I illustrated the distinction between a game's \state{States} and \state{Rules}, as well as the relationship between them using the example of \emph{rock-paper-scissors}. I then considered the important of this relationship via the problem of context sensitivity, which I explored using the case study of how \emph{The Legend of Zelda: Ocarina of Time} handles similar problems. 


\subsection{Visual Analysis of a Game}

\subsubsection{Introduction}
In this section, I provide a framework for analyzing a game's state information based on the its visual output using the constructs detailed in the previous sections. With this framework, I argue in the next section that a user who is aware of a game's external logic (the 'rules' a player intuitively learns via playing the game) can wield this visual analysis in building their own useful state model of the game.    

\subsubsection{Context Sensitivity and \rs}
As detailed in the previous section, we recognize the importance of context sensitivity in the design of \rs, which is tasked with understanding a game's 'rules' and 'contexts' just as a real-world player is. Unlike a real human, however, \rs cannot leverage the semantic power of the human mind. This means that visual concepts familiar to the average human---the shape of Pac-Man, or what color Mario's hat is---ultimately boils down to a collection of mere binary $1$s and $0$s for a computer. 

Consequent to this fact is the first problem we must solve: 

\newtheorem{q1}{Problem}
\begin{q1}
    How can \rs achieve the same level of context awareness that a human being reaches when visually analyzing a game? 
\end{q1}

To answer this problem, we must sub-divide it into two: the analysis of a game's \state{States} and \state{Rules}.

\subsubsection{Analyzing Game State}

Analyzing a game state is arguably the easier task, so we will begin our work here. 

\paragraph{Differentiating State}

To analyze a state, a player must be aware of what makes it \emph{distinct} from other states. Usually, this is done through some obvious visual cues. A game menu, such as a pause screen, is generally designed to look different than the in-game experience---clearly signaling to players that the game is not currently active. In a similar vein, the unique shades of blue and distinct green coral patterns of \emph{Super Mario Bros'} underwater levels indicate that a different play experience is to be expected. 

And so, in order to understand a game's active state, the player must:

\begin{enumerate}
    \item Look at the game.
    \item Detect the different objects that constitute the elements on-screen. 
    \item Categorize these objects into known entities.
    \item Relate this assortment of entities to a known \state{State}. 
\end{enumerate}

Logically then, we must translate this process into one that is computer-friendly.

\subparagraph{Looking at the Game}
Thankfully, the task of \textbf{1.} is simple enough. The current visual data of a game (hereafter referred to interchangeably as the game's active \emph{frame}) is easily copied from the buffers of memory which display it on-screen. We can then send this copy of the frame to whatever computer software we please.

\subparagraph{Object Detection}
Unfortunately, it is at step \textbf{2.} that we hit our first major roadblock. Or, to be more accurate, \emph{would} have hit a roadblock just a handful of years ago. 

As discussed in section 2, the history of computer-automated object detection is as long and storied as the problem is complex and layered. Work in the field has generally focused on specific identification tasks with a well-known utility---such as the detection of human faces or cancer cells. Technologies and algorithms powering object detection were often bespoke and incredibly arcane, despite their limited use-cases. Only in recent years has the field matured into a respectable domain, owing mainly to modern AI and Machine Learning discoveries. Consequently, the identification of objects via visual input is now a fairly trivial task given some basic programming knowledge. Efficient cross-platform detection algorithms such as YOLO (\emph{You Only Look Once}) have popularized object detection in a wealth of fields thanks to their relative ease-of-use\cite{wang2022yolov7trainablebagoffreebiessets}. 

\subparagraph{Object Categorization: \emph{The Journey}}
There is yet another problem, however. \textbf{3.} stipulates that we not only detect the \emph{existence} of objects, we must also \emph{categorize} these detected object into known entities. In other words, we must attach some sense of semantic meaning to these objects. In other words, the machine must not only recognize a collection of pixels as being identical from frame-to-frame, but must also recognize those pixels as 'Mario.' 

A purely automated solution to this problem is, unfortunately, outside the realm of this thesis (and likely modern science). There are certainly examples where an advanced, general computer vision algorithm may be able to recognize a popular entity like 'Mario'---who it has likely seen thousands of times within its training data---but this approach is not scalable for the domain of \emph{all} or even \emph{most} video games. 

In reality, these algorithms recognize popular gaming characters through many means external to games---advertisements, movies, t-shirts, etc. Even if games were a common aspect of training data, the bespoke nature of games means training off of one game will not guarantee success in another. To put it simply: being able to tell apart Mario, Luigi, and a \emph{Goomba} will do no good in understanding the difference between Cloud, Tifa, and a \emph{Tonberry} in \emph{Final Fantasy VII}.


\subparagraph{Object Categorization: \emph{The Answer}}
Consequently, \rs requires human intervention to overcome this problem. Even though \rs would not likely function using a general machine learning model, it is very easy to obtain good results using a \emph{specialized} model instead. Specialized detection models augment a model (pre-trained on general data) with new specific training examples provided by the implementor---200 images of the target game, for instance. Using this new data, the model can acquire an understanding of the target game after only a modicum of training---allowing for the reliable detection and categorization of in-game objects.

Of course, this process is not without its difficulties. For one thing, the act of providing image data to a machine learning algorithm is fairly laborious. Each image much be accurately annotated with labels describing the relevant objects contained within the image---requiring each picture to be manually edited and reviewed by a human being. On top of this, the exact number of annotated images required for good results is not consistent---with more complex games generally requiring larger datasets. Moreover, the \emph{quality} of the provided data is also crucial. Images of a low fidelity, as well as poorly annotated data, will often yield lackluster model behaviors.

Despite all of these issues, the act of annotating training data has become commonplace in modern machine learning. There exist a wealth of tools---such as the open source \emph{Make Sense} project\cite{make-sense} or the corporate projects of companies such as \emph{RoboFlow}\cite{Roboflow_Supervision}---which streamline this process significantly. Moreover, the act of image acquisition is often trivial, merely demanding an implementor record a few minutes of active gameplay. The implementor could then simply pull a still frame from every few seconds of the video to use as data for annotation. 

Given a few hours and some basic manual labor, a machine learning algorithm can be augmented with enough game-specific knowledge to enable the high-quality categorization of objects via computer vision---solving the hurdles of \textbf{3.} 

Despite this, it should also be noted that machine learning, computer vision, and AI are not at all required to accomplish this task. A programmer willing to perform these operations manually---to design their own program which attaches semantic value to objects---can achieve the same results. \rs is ultimately indifferent to the exact approach to this problem---it only asks for good semantic information irregardless of its origin.

\subparagraph{Relating to State}
Now that we have all of this information---the game frame, the objects within the frame, and the semantic meaning of those objects---we must now perform the final and most crucial step: building a distinct mapping of this information set to one \state{State}. This process also requires human intervention, as it yet again demands the input data with additional semantic meaning. 

This is the first area where \rs sees use---providing the format and framework for building these semantic relationships. This process---known as the \emph{Design Phase}---is detailed in a later section. For now, it is sufficient to say the implementor provides \rs with information about the game's semantics.

\paragraph{Summary}

With these challenges overcome, \rs is now able to 'see' and 'understand' the state of the game much like a user can---translating visual information (the game's current frame) into meaningful semantic information (the game's current \state{State}). 

This puts us one step closer, but we are still missing something: an understanding of the \state{Rules} which underline the active \state{State}.

\subsubsection{Analyzing \state{Rules}}
Compared to the lengthy process required for \state{State} analysis, \rules are deduced in a far simpler fashion. One may observe that the structures and consequences of \rules are often non-visual and potentially abstract in nature, meaning our computer vision approach is yet again trumped by semantic and context sensitive information. But, as detailed in previous sections, even a game's abstract \rules are generally intuited by the player. For example, a player might understand that a switch flipped in one room of a game's level will have some effect on an entity in a different, yet unseen area. 

And so, we require human intervention yet again through the \state{Design} phase of \rs's implementation process. The implementor must provide some simple input which describes the given \rules of a game's state in formal logic---which serves as the 'language' the computer is capable of understanding.

\subsubsection{Conclusion}
In this section, I dissected the visual experience of analyzing a game from the player's perspective. From this, I built a set of requirements that \rs must satisfy in order to successfully 'understand' the game like a real human player would based on the same visual information. I then elaborated the challenge presented by each requirement, as well as how \rs overcomes these challenges in order to successfully intuit a game based using the same visual input a human player uses. 

\section{Structural Outline}

\subsection{Overview}

\subsubsection{Introduction}
In this section, I provide an overview of the three 'layers'---\vision, and \decision, \audio---which constitute \rs's structure. 

\subsubsection{Turning Theory into Practice}

Now that we have explored the concepts needed for understanding \rs, it has come time to finally explain the structure of the framework in detail. It should be clear by now that \rs reaches into several areas of study for its theoretical basis---performing tasks in numerous domains to achieve our goals. This multi-modal approach makes explaining \rs's internal workings difficult. In order to ease this process, I will subdivide \rs into three key 'layers.' Each layer is tasked with solving problems in different domains and consequently has its own distinct traits and behaviors. The layers are:

\begin{description}
    \item[The Vision Layer] Converts a game's visual information in formal game logic. 
    \item[The Decision Layer] Converts formal game logic into \rs's own internal logic.
    \item[The Audification Layer] Converts \rs's internal logic into spatial audio, which is output for the end user.
\end{description}

It is hopefully clear how these three layers interact---and how the \vision layer cascades downwards towards the \audio layer, starting with visual data input and eventually generating audio data output. I will now elaborate on each layer in detail. 

\subsection{Vision Layer}

The \vision layer can be imagined as the starting point of \rs's execution loop. In it, the game's frame data (its visual information) is provided to a computer vision algorithm. The frame can be provided in a host of ways. 

A programmer with direct access to the game's engine may choose to copy the framebuffer directly from its original in-engine source to \rs. Those without access to the game's internal logic may instead opt for a simple screen capture implementation, which essentially sends screenshots of the game window to the computer vision algorithm. This may seem inefficient at first glance, but all major operating systems have support for screen capture built-in at the kernel level, which makes this option a very feasible and attractive route for users with relatively modern computer hardware\footnote{See \texttt{PrintWindow} in the Windows API\cite{PrintWindow} and \texttt{XGetImage} in the X11 API\cite{XGetWindow} for modern examples.}. 

This algorithm is responsible for the detection and classification of game entities contained within the frame. The algorithm may achieve object detection by any means desired by the implementor (machine learning, corner detection, template matching, etc.). 

This same freedom is not extended to the act of object classification, however. The implementor must provide the \vision layer with object semantics. These semantics associate a detected object with a specific entity type---a mushroom in \emph{Mario}, a slime in \emph{Dragon Quest}, or a potion in \emph{Zelda}, for example. Semantic associations can be included in any way the implementor pleases---training data for a machine learning model, custom programming logic, etc. 

The \vision layer is, through this processes able to translate a visual image of the game into semantic information about the game's active \state{State}---abstracting visuals into game logic. 

To summarize:

\Summary[]{

{\begin{center}
    \state{Vision Layer}
\end{center}}

\begin{description}
    \item [Receives input from:] The game's graphical frame buffer.
    \item[Input:] An image of the game's current on screen graphics (\emph{frame data})
    \item[Behavior:] Uses a computer vision algorithm to perform object detection. Then uses semantic information provided by the implementor to classify these objects into game entities.
    \item[Output:] Semantic information about the game's current state---as defined by the entities currently visible on screen. 
    \item[Sends output to:] The \decision layer.    
\end{description}
}


\subsection{Decision Layer}

The \decision layer is arguably the most complex layer, at least upon initial inspection. After receiving semantics about the game's current frame from the \vision layer, this layer is tasked with translating these semantics into the 'course of action' \rs takes in response to this frame. In other words, this layer is where \rs's \emph{decision making} occurs. In essence, this layer can be considered the 'heart' or 'core' of \rs---as it takes place entirely within its own confines.

Of course, a program can only make decisions imbued in it by its creator---and \rs is no different. In reality, the implementor supplies \rs with a collection of simple conditional statements that translate into the range of possible decisions \rs is capable of making. For example, a supplied conditional statement may look something like this:



\begin{algorithm}
    \KwData{Game semantics S  from the \vision layer }
    \KwResult{Procedural calls to generate \audio objects}
    \If{{\bf Entity} 'Mario' is in S {\bf and} {\bf Entity} 'Mushroom' is in S}{
     \If{distance(Mario, Mushroom) $\leq$ dMushroomCollision}{
        play mushroom power-up SFX;
      }
    }
    \caption{A simple semantic \decision}
   \end{algorithm}

In this simple case, the semantic data was used to check on existence of two entities---defined semantically as 'Mario' and 'Mushroom.' If these two entities are found to be on-screen at the same time, another check is made to see if they are close to each other---within a given distance $d$, as defined by the implementor. If so, \rs is told to play a specific spatial audio cue---a 'power-up' sound effect, which indicates what is already clear to a sighed player: that Mario has collided with a mushroom, giving him a power-up effect.

Of course, this is a trivial example---as most games already provide auditory feedback for simple actions like powering up. But the extensible functionality of \rs allows for the translation of many non-auditory features into sound. For example, the position of a specific entity could be mapped from an physical point in game-space to a spatial point in the user's audio-space---generating sounds around the user's head to indicate their position in a clear, intuitive fashion.

\subsubsection{Utilities and Extensions}

The totality of these features allow \rs---under the imperative command of the implementor---to make active \state{Decisions} about what sound objects to generate in a given situation. More that, \state{Decisions} also allow the implementor freedom in controlling the 'How?' 'When?' and 'Where?' of \audio. More generally, the implementor is able to perform a host of non-audio related tasks using this functionality as well---updating state, entity, and meta attributes, generating other forms of user-understandable feedback, etc. 

To achieve this, several simple constructs are provided by \rs to streamline the implementation of \state{Decisions.} For example, a $distance(x, y)$ function was used above to quickly calculate the distance between two entities. This can be applied to any given entity in \rs, as each entity's semantic is bundled with its screen-space cartesian location. Other foundational functions---such as a $amount(x)$ function (for calculating how many instances of an entity type are on-screen) and a $size(x)$ function (for quickly figuring the scale of an entity)---are also provided.  

To summarize:

\Summary[]{

{\begin{center}
    \state{Decision Layer}
\end{center}}

\begin{description}
    \item [Receives input from:] \state{Vision Layer}
    \item[Input:] The game's current state, as described in semantic terms defined by the implementor.
    \item[Input:] Conditions and imperatives as defined by the implementor. 
    \item[Behavior:] Evaluates the given semantic data against the provided conditions. If a condition evaluates to \texttt{true}, the associated imperative is executed.
    \item[Output:] Various commands. Primarily the command to generate spatial audio objects, with behaviors defined by associated imperatives.
    \item[Sends output to:] The \audio layer.    
\end{description}
}




\subsection{Audification Layer}

Finally, we arrive at the \audio layer, which is responsible for the generation of spatial audio objects that are played back to the end user---enabling them to 'hear' the information usually 'seen' in playing a game.

This layer can be considered the simplest of the three, as most of its duties are delegated to any number of third-party audio libraries which exist outside of \rs. As mentioned in previous chapters, spatial and 3D audio technology has matured to the point of trivializing the generation of dynamic spatial audio---like in our use-case here. Consequently, the actual specifics of this layer are left up to the implementor---provided they accurately generate the audio as stipulated by the \decision layer.

The comparatively trivial nature of this layer might leave one wondering exactly why it is defined as a distinct layer at all. This is because \audio serves a clear and distinct purpose within \rs (\emph{it is the system's name sake, after all!}). \audio also has its own set of unique behaviors and exists in a different domain (audio and sound) when compared to the previous two layers (vision and internal game semantics). Lastly, \audio produces our final and ultimate output---the sounds being played back to the end user. For these reasons, I have decided to distinguish it as its own pillar of \rs's structure.  

To summarize: 

\Summary[]{

{\begin{center}
    \state{Audification Layer}
\end{center}}

\begin{description}
    \item [Receives input from:] \state{Decision Layer}
    \item[Input:] A collection of spatial audio objects, as well as rules for how to instantiate them.
    \item[Behavior:] Generates spatial audio based on the conditions provided as input.
    \item[Output:] Spatial audio, which is outputted to the end user's headphones via a framework supporting 3D and spatial audio technology. 
    \item[Sends output to:] Spatial/3D Audio playback libraries, the operating system, and eventually, the end user.
\end{description}
}




\subsection{Summary}

In this section, I provided a in-depth analysis of \rs's structure. I subdivided the system into three core layers which have unique behaviors spread across different problem domains. These layers---\vision, \decision, and \audio---ultimately work as a pipeline: taking a game's visual information, converting it into semantic data about the game's state (\vision), converting those semantics into actionable imperatives (\decision), and finally converting some of those imperatives into spatial audio objects (\audio). Figure X [TODO: ADD!] summarizes the entire structure of \rs.





\section{Implementation Process}
In this section, I outline \rs's principal operational process---the \plan, \train, \design, \exec loop. To conclude, I consider the structure's strengths and weaknesses.        

\subsection{Overview}
In the previous section, I detailed \rs's internal structure, and provided some hints as how an implementor interacts with it. But this structural overview paints a theoretical picture of software, not a full picture of how it can be used in the real world. In this section, I go into specifics on how an implementor applies \rs and an end user actually interacts with the software. To do so, I conceptualize the \plan, \train, \decision, \exec process loop---which serves as a useful guide to understanding how \rs may be implemented into a given target game. 

\subsection{Design}

And so, we begin with the first phase---\design---which occurs before a implementor even touches a computer. First, the \imp must come to understand the target game's internal logic and \design the semantic relationships which \rs will use for operation.

This phase may appear simple at first glance---as it requires no direct programming or technical applications---but is actually the most difficult phase. To borrow the old Sagan aphorism:

\begin{quote}
    
\emph{"If you wish to make an apple pie from scratch, you must first invent the universe."}
    
\end{quote}

In order to make the 'apple pie' that is \rs's implementation into a target game, the \design phase must first 'invent' the universe which contains it. The quality of the entire implementation depends upon a successful \design---earning this phase the title of 'most critical.' If the foundational \design is poor, then the following phases will no doubt also yield poor results. 

This phase involves several steps, which will we now investigate in detail.

\subsubsection{Analyzing Game Semantics}

First, the \imp must come to terms with their target game---the, \state{States}, \rules, and \state{States} that it consists of. For a simple target game---such as a classic arcade game or early home-console title---this process is fairly trivial. The \imp may break the game down into a series of entities, \state{States}, and \rules which apply to each state.

\paragraph{\emph{Pac-Man:} A Worked Example }

For example, one may observe \emph{Pac-Man} consists of only a few key states---an \state{attract} screen that plays before the user starts the game, a clear \state{opening} before which plays before each life or level begins, the \state{normal} gameplay cycle, the special \state{power-up} cycle (where ghosts are vulnerable), a \state{win} phase when a player successfully completes a level, a \state{death} sequence which occurs whenever the player collides with a ghost and a \state{game-over} phase. One may also easily observe the few entities which comprise the game---\emph{Pac-Man}, all four of the \emph{Ghosts}, the standard \emph{Pellet}, the \emph{Power Pellet}, the several \emph{Items} which occasionally appear on-screen, etc.

\subparagraph{Designing Transitions}

Using these two groupings---the \state{States} and the Entities---one may finally attach \rules to each state, as well as the transitions between them. For example, the \textbf{PRESS START} text which appears during the \state{attract} phase could be classified as an entity. The presence of this entity on screen must mean the game is currently in the \state{attract} phase. The sudden appearance of other entities on screen---such as the \emph{Pellets} or \emph{Pac-Man} will indicate a player has started the game---taking us to the \state{opening} of the game. Observations like these are how an \imp builds the transitional function $T$ for each \state{State}. It should be noted that these transitions serve as a special kind of \state{Rule} for each \state{State}---one dictating how and when to move between \state{States}.

\subsubsection{Desinging \state{Rules}}

\paragraph{State Rules}

Now that we are able to move between \state{States}, we must also define the \rules which comprise each one. For some \state{States} such as \state{attract} or \state{game-over}, this is a trivial operation---usually requiring only transitional \rules. For the advanced \state{States}---the ones which usually comprise the core gameplay loop---a more careful treatment is required. An \imp needs to intelligently organize in-game actions into useful logical constructs. Some \rules are simple---such as checking if the distance between \emph{Pac-Man} and any of the \emph{Ghosts} is less than some small value, which would indicate a need to transition to \state{death.} Other rules may have more nuance, such as understanding when the player achieves an extra-life after acquiring enough score points.
One may notice that I did not detail any sort of \state{extra-life} state in the previous section. Nor have I detailed the \rules needed in ascertaining a transition from the \state{normal} game state to the eventual \state{death} state. 

This is because we have only considered half of the \state{States} and \rules of the target game. We now consider the other half---those belonging to each Entity. 


\paragraph{\state{Entity} \state{Rules}}
Like the game itself, each entity has its own collections of \state{States} and \rules which govern it. These constructs describe behaviors an entity has over its lifetime. To return to the example above, we can consider the behaviors (i.e. \state{States}) of the \emph{Pac-Man} entity.  

As most readers are no doubt aware, \emph{Pac-man} is famous for his voracious appetite---\state{eating} pellets as he traverses the game world. He also enters a \state{power-up} state, where he is able to destroy ghosts. As alluded to above, he also \state{dies} whenever he collides with a \emph{Ghosts} while not \state{powered-up}. These three states---\state{eating}, \state{power-up}, \state{death}---cover \emph{Pac-Man's} obvious behavior, but there are less opaque \state{States} we must also consider. How can we reconcile this oversight in our \design? 

\paragraph{Transient States}

For example, \emph{Pac-Man} is able to gain additional lives if a specific score is achieved---a behavior not yet recognized by our \state{State} model. To account for this, we must also consider the notion of \emph{transient} \state{States}, which generally persist for only a single frame of in-game time. To illustrate this point further, consider the case when \emph{Pac-Man} touches an item object. The player is awarded a score boost, which has in-game consequences. Naturally, the player should be notified about this special occurrence. Of course in this specific case the game provides several kinds of feedback---removing the item object from screen, replacing it with a numeral score visual, and playing a special sound effect. 

But these transient events are not always so clearly signaled. A player might receive audio feedback when \emph{Pac-Man} consumes an item, but they will not hear any indication of when that item first spawns on screen---obscuring key \state{State} information that a sighted player has easy access to. And even though item consumption has audio feedback, it does not paint the full picture---obscuring the score value of the item, which is only signaled to the player visually. 

Through these considerations, it becomes clear that transient \state{States} also play a key part in \design. Thankfully, most of these problems are solved by the inclusion of audio cues indicating transient events to the player. For example, the \imp could design special audio cues that are generated whenever a item first appears on-screen---creating a unique sound that differentiates each item type. On top of this, a unique pitch could be generated each time an item is consumed---the frequency of the pitch being related to the item's score value. 

\paragraph{State Delegation}
The existence of Entities and transient \state{States} presents one more interesting problem: the \emph{delegation} of responsibilities between Entities. In the previous example, we considered the interactions between \emph{Pac-Man} and the various items that occasionally appear on-screen. I generally implied that \emph{Pac-Man} is responsible for maintaining the relevant \state{State} information about these interactions. But is that really the case?

A state machine (at least as we define them here) can only exist in one state at a time. This means that \emph{Pac-Man} would need to leave his \state{normal} state in order to respond to special events such as when an item is spawned or consumed. This is generally unproductive to the overall experience. For example, if a important sound was being generated every frame by \emph{Pac-Man}---perhaps a sound indicating his on-screen position---the sound would need to be temporarily paused in order to handle the transient \state{State} change. This may be a non-issue in most cases, but it is easy to imagine situations where even short lapses in the consistency of key information spells trouble. 

And so, the \imp must also take great care in delegating the responsibilities and behaviors of each entity. In this case, the aforementioned tasks could instead be handled by the items themselves---leaving \emph{Pac-Man} free to perform other \state{States} which are more relevant to the overall experience. An item could have states like \state{spawned} and \state{consumed}, which handle the situations described previously. On top of this, an item may also have a \state{normal} state---playing a spatial sound which describes it current location relative to \emph{Pac-Man}.


\paragraph{Summary}

We have wandered down a bit of a semantic rabbit hole via our \emph{Pac-Man} example, so let allow us to return to the big-picture. In the \design phase, the \imp considers the characteristics of the target game and designs a 'model' of the game. The model comprises the \state{States}, \rules, and Entities of the game. 

It should be stated yet again that this design has not yet rendered a single line of code. The model instead serves as the \emph{guidelines} which drive the following steps of the implementation process. 

To summarize:   

\Summary[]{

{\begin{center}
    \state{Design Phase}
\end{center}}

\begin{description}
    \item [Requires:] A target game. Game analysis skills.
    \item [Involves:] The design of a model replicating the target game's characteristic behaviors. This model consists of
    \begin{description}
        \item[Game States] A list of states that comprise the entire game experience
        \item[Game State Rules] A collection of rules that characterize what occurs during each game state. This also includes rules for how to transition between game states.
        \item[Entities] A collection of entities that comprise all of the actors within the game.
        \item[Entity States] A collection of states that describe the actions each entity is capable of.
        \item[Entity State Rules] The rules which characterize each Entity state's behaviors---how an entity dies, what happens when it is collided with, etc.      
    \end{description}
    \item [Purpose:] Generates a design plan that serves as the guide for following steps.
    \item [Notes:] This model is purely 'on-paper.' No code or software is required for this phase. 
\end{description}

}


\subsection{Training}

I remarked at the start of this section that we were finally going to put theory into practice---a promise we have yet to fulfill. Let us amend that now with the \train phase. In this phase, the \imp is tasked with choosing and implementing a real means of analyzing the game's frame data. There are many different ways of accomplishing this task. We shall consider a few of primary methods now.   

\subsubsection{Computer Vision - Machine Learning}
As we have no doubt made clear by now, virtually every useful method falls somewhere under the umbrella of computer vision. In recent years, this has meant the implementation of machine learning based object detection algorithms such as \emph{YOLO}. In this case, the implementor is tasked with generating the relevant training data needed to tune their machine learning model to the target game. As detailed in the previous section, this generally involves the creation of \train data which is used to train the model (hopefully making clear the provenance of this phase's name).

It is hopefully clear to us now why the \design phase precedes \train. As discussed in earlier sections the \imp must annotate images of the games state in order to create a training data set. Each photo must clearly and consistently label game entities that appear within the frame. Without a clear \design, the \imp risks creating inconsistent annotations and a mediocre dataset. This would likely yield subpar training results and consequently a lackluster computer vision algorithm---impeding future steps.

At any rate, we suppose that the \imp manages to successfully train a machine learning model. This model is capable of receiving a image of the game as input and outputting a list of entities which appear in the image---each one annotated with distinguishing information like the entity type and their position on screen. This suffices to satisfy the \train process. 

\subsubsection{Computer Vision - Alternative Methods}
But machine learning is not the only answer to this problem. Depending on the target game, other methods may prove to be easier to implement and more efficient in handling the \train requirements. Some examples include the previously discussed techniques of template matching and corner detection. Each uses a simple algorithm that performs a comparatively naive analysis of the frame, but may still yield successful results at a far more efficient speed than machine learning alternatives.

There are still drawbacks to these alternatives, even when they prove to be successful. The appeal of machine learning is its ease of implementation into a wide variety of problem domains---a benefit not extended to alternative solutions. Instead, these methods demand bespoke software solutions that are often byzantine in nature. 

An implementation of a corner detection algorithm (which we will demonstrate in the following section), for example, also demands custom semantics for translating detected corners into objects. This translation process is extremely specific to each game---and may involve painful pixel-level calculations that are often unreliable. Other methods like template matching allow the \imp some reprieve, but ultimately come off as an incredibly naive form of machine learning. However, there may still be cases where these simple approaches have utility, as we will see later.

\subsubsection{Summary}
In essence, the \train phase is where we first put theory into practice---choosing and implementing a computer vision algorithm capable of transforming a game image into a collection of well-defined entities which are processed in later stages. \rs aims to generalize this process by putting very few restrictions on the \imp, who may choose any current (or future, from the perspective of this writing) techniques they please. 

To summarize:

\Summary[]{

{\begin{center}
    \state{Training Phase}
\end{center}}

\begin{description}
    \item [Requires:] A computer vision algorithm. Semantic data as created in the \design phase.
    \item [Involves:] The implementation of a algorithm capable of translating a game's frame data into a list of entities which appear on-screen.
    \item [Purpose:] Translates visual information into game semantics---serving as the \vision layer described in previous sections. 
    \item [Notes:] Machine learning models are generally used, but the \imp is free to choose any method they please, provided it generates the correct output for later phases. 
\end{description}

}


\subsection{Implementation}



\subsection{Play}


\subsection{Summary}

\section{Considerations}



%%% IMPLEMENTATION

\chapter{Sample Implementation}
\section{Languages, Tools, and Frameworks}
    \subsection{Python}
    \subsection{OpenCV}
    \subsection{Qt}
\section{Choosing a Target Game}
\section{Pong: A Worked Demonstration}

%%% EXPERIMENTS

\chapter{Experiments} 
\section{Pong Demonstration}
\subsection{Results}

%%% CONCLUSION
\chapter{Conclusions}
\section{Limitations}
\section{Future Work}

\section{Thanks}

% \bibliographystyle{IEEEtran}
\printbibliography

\end{document}

